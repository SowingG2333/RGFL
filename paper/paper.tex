\documentclass[conference]{IEEEtran}
\IEEEoverridecommandlockouts
\usepackage{cite}
\usepackage{amsmath,amssymb,amsfonts}
\usepackage{algorithmic}
\usepackage{graphicx}
\usepackage{textcomp}
\usepackage{xcolor}
\usepackage{mathtools}
\usepackage{amsthm}

% Define theorem environments if needed for the proof section
\newtheorem{theorem}{Theorem}
\newtheorem{lemma}{Lemma}
\newtheorem{definition}{Definition}
\newtheorem{assumption}{Assumption}

% Define math operators and commands used in the proof
\newcommand{\HC}{\mathcal{H}}
\newcommand{\FC}{\mathcal{F}}
\newcommand{\E}{\mathbb{E}}
\newcommand{\Prob}{\Pr}
\newcommand{\argmax}{\operatornamewithlimits{argmax}}
\DeclarePairedDelimiter{\abs}{\lvert}{\rvert}
\DeclarePairedDelimiter{\round}{\lfloor}{\rceil}

\def\BibTeX{{\rm B\kern-.05em{\sc i\kern-.025em b}\kern-.08em
    T\kern-.1667em\lower.7ex\hbox{E}\kern-.125emX}}

\begin{document}

\title{Reversing Gresham's Law: Exploiting Parasitic Dependencies for Free-Rider Detection in Blockchain Federated Learning}

\author{
    \IEEEauthorblockN{Donghang Duan}
    \IEEEauthorblockA{\textit{Yingcai Honor College} \\
    \textit{University of Electronic Science and Technology of China}\\
    Chengdu, China \\
    }
}
\maketitle

% \begin{abstract}
% \end{abstract}

% \begin{IEEEkeywords}
% \end{IEEEkeywords}


\section{Problem Definition}
\label{sec:problem_definition}
In this section, we define our system model first. Then we formally present the problem to be solved in this paper.

\subsection{System Model}
\label{sec:system_model}
In this paper, we consider a blockchain-based federated learning system operating over multiple rounds $t \in \{1, 2, \dots, T\}$ in an environment potentially containing untrusted participants.
The system involves a requester $\mathcal{R}$ and a set of $N$ participants, denoted as $\mathcal{P} = \{p_1, p_2, \dots, p_N\}$.

The participants $\mathcal{P}$ are partitioned into two disjoint sets: honest clients $\mathcal{H} = \{h_1, h_2, \dots, h_{N_h}\}$ and free-riders $\mathcal{F} = \{f_1, f_2, \dots, f_{N_f}\}$, where $N = N_h + N_f$.
Honest clients genuinely train the model on their local dataset $\mathcal{D}_i$, while free-riders who lacks data or unwilling to expend resources, attempt to obtain rewards and the final model by submitting fabricated updates.

To detect and remove the free-riders, the system employs a reverse auction mechanism coupled with a reputation system. 
In each round $t$, participants $p_i \in \mathcal{P}$ will be asked to submit a bid to the requester $\mathcal{R}$. 
A bid can be denoted as ($P_i^t$, $R_i^t$), where $P_i^t$ represents the promised contribution to the increase in model performance and $R_i^t$ denotes the reward that the participant $p_i$ expects.
The requester $\mathcal{R}$ then selects a subset $C_t \subseteq \mathcal{P}$ of size $M_t$ by ranking the bids based on cost-effectiveness defined by the ratio $\rho_i^t = P_i^t / R_i^t$.
In this way, the system forms a reverse auction where participants compete by offering better cost-effectiveness ratios to be selected.

Each participant $p_i$ possesses a reputation score $r_i^t$, initialized to $r_0$. 
After the selected participants $C_t$ submit their model updates, the requester $\mathcal{R}$ publishes a test dataset $\mathcal{D}_{\text{test}}$ and verifies the updates' performance on it, which can be checked on the blockchain.
The reputation $r_i^t$ of participants in $C_t$ is updated based on this verification outcome, while participants not selected maintain their reputation. 
If the verification result aligns with the promised contribution $P_i^t$, the participant is rewarded with $R_i^t$ and its reputation is increased.
Otherwise, the participant is penalized with a decreased on its reputation.

We assume the data distribution across participants is non-IID (Independent and Identically Distributed): participant $i$'s private dataset $\mathcal{D}_i$ may follow a distribution different from others and $\mathcal{D}_{\text{test}}$.
The overall goal is to remove all the free-riders from the system while optimize the final global model's performance on $\mathcal{D}_{\text{test}}$ after the least number of rounds $T$.

\subsection{Problem Formulation}
\label{sec:problem_formulation}
Free-rider $j \in \mathcal{F}$ aims to maximize its reward, while minimizing its computational and data costs.
In our paper, we assume that the free-riders don't have any data. They attempt to generate fabricated gradients using an advanced mechanism $\mathcal{A}$, which is based on the Adam (Adaptive Moment Estimation) optimizer, to mimic honest participation.
According to previous global models' gradients, the mechanism $\mathcal{A}$ can potentially generate fabricated gradients indistinguishable from the gradients generated by honest clients on statistical grounds, exploiting the non-IID nature of data in federated learning.

However, the mechanism $\mathcal{A}$ fundamentally relies on fresh and real gradient updates from normal clients to generate fabricated gradients that are effective in improving the model performance.
If the available history of model updates is out of date or lack of amount, the fabricated gradients' update direction will significantly deviate from the true gradient direction, potentially leading to a decrease in the model performance and revealing the free-rider's nature.

The Adam-based mechanism $\mathcal{A}$ is defined as follows:
\begin{equation}
    \mathcal{A}(\theta^t, \theta^{t-1}) =
    \begin{cases}
        \theta^t + \delta & \text{if } t = 0 \\
        \theta^t + \Delta\theta_{\text{Adam}} & \text{if } t > 0
    \end{cases}
    \label{eq:adam_adversary}
\end{equation}
\begin{align}
    g_t &= \theta^t - \theta^{t-1} \\
    m_t &= \beta_1 m_{t-1} + (1-\beta_1)g_t \\
    v_t &= \beta_2 v_{t-1} + (1-\beta_2)g_t^2 \\
    \hat{m}_t &= \frac{m_t}{1-\beta_1^t} \\
    \hat{v}_t &= \frac{v_t}{1-\beta_2^t} \\
    \Delta\theta_{\text{Adam}} &= -\eta\frac{\hat{m}_t}{\sqrt{\hat{v}_t} + \epsilon} \label{eq:adam_update}
\end{align}
\vspace{4em}

Where:
\begin{itemize}
    \item $\delta \sim \mathcal{N}(0, \sigma^2)$ is a random gaussian noise.
    \item $m_t$ and $v_t$ are the first and second moment estimates, initialized as $m_0 = 0, v_0 = 0$.
    \item $\beta_1, \beta_2 \in [0,1)$ are exponential decay rates for the moment estimates.
    \item $\hat{m}_t$ and $\hat{v}_t$ are bias-corrected moment estimates.
    \item $\eta$ is the learning rate used by the free-rider's mechanism.
    \item $\epsilon$ is a small constant to prevent division by zero.
\end{itemize}

\vspace{1em}
What's worth mentioning is that the term $g_0$ cannot be computed in the first round ($t=0$) as it requires $\theta^0$ and $\theta^{-1}$, hence the free-rider submits the original model $\theta^0$ with a random noise $\delta$ added to it.

We can prove that the free-riders will be removed from the system due to its greedy strategy, no matter what advanced mechanism they use to generate fabricated gradient updates.
Therefore, we aim to detect and remove the free-riders from our system with least reward cost while ensuring that the honest clients won't be misjudged as free-riders.

To achieve this, we formulate a multi-objective optimization problem. 
Let $\mathcal{X}$ represent the set of tunable system parameters, and $\mathbb{X}$ be the feasible search space for these parameters $\mathcal{X}$.

We define the following performance metrics based on a full simulation run of the system with $\mathcal{X}$:
\begin{itemize}
    \item $T_{elim}(\mathcal{X})$: The total number of rounds required until all free-riders $f \in \mathcal{F}$ have their reputations $r_f^t < r_{thresh}$.
    \item $C_{total}(\mathcal{X})$: The cumulative sum of rewards $R_i^t$ disbursed by the requester $\mathcal{R}$ to all selected participants $p_i \in C_t$ up to round $T_{elim}(\mathcal{X})$.
    \item $N_h$: The total number of honest clients in the set $\mathcal{H}$.
    \item $N_{h,elim}(\mathcal{X})$: The number of honest clients $h \in \mathcal{H}$ whose reputations $r_h^t$ fall below $r_{thresh}$ by round $T_{elim}(\mathcal{X})$.
    \item $FPR(\mathcal{X}) = \frac{N_{h,elim}(\mathcal{X})}{N_h}$: The False Positive Rate, representing the misjudgment rate of honest clients.
    \item $PFM_{final}(\mathcal{X})$: The performance of the final global model $\theta^{T_{elim}(\mathcal{X})}$ evaluated on the test dataset $\mathcal{D}_{\text{test}}$.
    \item $PFM_{min}$: A predefined minimum acceptable performance for $Acc_{final}(\mathcal{X})$.
\end{itemize}

The optimization problem is to find a set of parameters $\mathcal{X}$ that achieve a Pareto optimal trade-off between minimizing the total reward cost and minimizing the misjudgment rate of honest clients, subject to maintaining a satisfactory final model performance:

\begin{equation}
    \begin{aligned}
    & \underset{\mathcal{X} \in \mathbb{X}}{\text{minimize}}
    & & \left( C_{total}(\mathcal{X}), FPR(\mathcal{X}) \right) \\
    & \text{subject to}
    & & PFM_{final}(\mathcal{X}) \ge PFM_{min}
    \end{aligned}
\end{equation}

Solving this multi-objective optimization problem will yield a set of Pareto optimal solutions. Each solution in this set represents a different balance between $C_{total}(\mathcal{X})$ and $FPR(\mathcal{X})$ that cannot be improved in one objective without degrading the other, while respecting the minimum model accuracy constraint. The final selection of parameters from this Pareto set will depend on the specific priorities and risk tolerance of the requester $\mathcal{R}$.

\section{Mathcal Proof of Convergence}
\label{sec:convergence_proof}

The proof of convergence unfolds in several dependent stages. We first establish that the selection size, $M_t$, converges to the actual number of free-riders, $N_f$. This convergence, driven by the bidding advantages of free-riders and the reputation dynamics, leads to a diminishing participation rate of honest clients in the selected set. Consequently, free-riders, who rely on fresh updates from honest clients, experience a degradation in their performance, causing their success probability to plummet. This sequence of events ensures their continued selection (as they exhibit large reputation changes, albeit negative) and triggers a rapid collapse of their reputations, ultimately leading to their elimination from the system.

\begin{lemma}[Convergence of Selection Size $M_t$]
\label{lemma:mt_convergence}
The selection size $M_t$ converges to the number of free-riders $N_f$ as $t \to \infty$.
\end{lemma}
\begin{proof}
We analyze the convergence behavior of $M_t$ by considering the extended reputation change metric $\delta r_i^t = r_i^t - r_i^{t-q}$, which tracks reputation changes over $q$ rounds. This metric is crucial for understanding the dynamics in both reputation growth and decline phases of free-riders. The value $k'_t = \argmax_j (\abs{\delta r}_j^{top} - \abs{\delta r}_{j+1}^{top})$ identifies the index after the largest gap in sorted absolute reputation changes, and $M_t$ is updated via $M_t = \round{\omega M_{t-1} + (1-\omega) k'_t}$.

In the early stages, when free-riders successfully mimic honest clients ($q_f^t \approx 1$), their promised contributions $P_f^t$ are high relative to their requested rewards $R_f^t$, leading to $\rho_f^t \approx \gamma \bar{\rho}_h^t$ with $\gamma > 1$ (Assumption 2). This results in a higher selection probability $p_f^t > p_h^t$. Consequently, free-riders are more likely to be in $C_t$ and experience reputation increases. The reputation changes for successful free-riders are positive:
\begin{equation}
    \delta r_f^t = r_f^t - r_f^{t-q} \approx \sum_{i=t-q}^t p_f^i \cdot \alpha
\end{equation}
Similarly, selected honest clients also experience positive reputation changes:
\begin{equation}
    \delta r_h^t = r_h^t - r_h^{t-q} \approx \sum_{i=t-q}^t p_h^i \cdot \alpha
\end{equation}
Due to $p_f^t > p_h^t$, free-riders are more consistently selected and accumulate reputation changes. When clients are ranked by $|\delta r_i^t|$ in descending order, the top positions are predominantly occupied by these $N_f$ free-riders. Thus, the largest gap in $|\delta r_i^t|$ values likely occurs after these $N_f$ clients, leading to:
\begin{equation}
    k'_t \approx N_f
\end{equation}

As the system evolves and the honest participation rate $\eta_h^t$ potentially drops (as will be shown in Section \ref{phase:hc_decline}), free-riders may begin to fail. Even during a transition period where some free-riders succeed and others fail, they, as a group, still constitute the majority of selected clients experiencing significant (positive or negative) reputation changes. The distribution of $|\delta r_i^t|$ is expected to maintain a significant gap after approximately $N_f$ clients, thus keeping:
\begin{equation}
    k'_t \approx N_f
\end{equation}

Eventually, when the honest participation rate is low enough such that $q_f^t \to 0$ for $t > T_2$ (as will be shown in Section \ref{phase:fr_degradation}), free-riders consistently fail upon selection. For a free-rider $f \in \FC$ selected in consecutive rounds, the reputation change over $q$ rounds becomes significantly negative:
\begin{equation}
    \delta r_f^t = r_f^t - r_f^{t-q} \approx -\sum_{j=0}^{q-1}\beta^{k_f^{t-j}} < 0
\end{equation}
The absolute value $|\delta r_f^t|$ grows exponentially with the number of failures $k_f^t$ since $\beta > 1$. Honest clients, if selected, have $|\delta r_h^t| \approx \alpha$ (if successful), but their selection probability $p_h^t$ becomes very low once free-riders dominate $C_t$. Unselected clients have $|\delta r_i^t| = 0$. Therefore, the absolute reputation changes of consistently failing free-riders dominate all others:
\begin{equation}
    \forall f \in \FC \cap C_t, \quad \abs{\delta r_f^t} \gg \abs{\delta r_h^t}, \abs{\delta r_i^t} \text{ for } h \in \HC, i \notin C_t
\end{equation}
When ranking clients by $|\delta r_i^t|$, the top $N_f$ positions will be occupied by these failing free-riders. The largest gap in this ranking occurs after this group, leading to:
\begin{equation}
    k'_t \to N_f
\end{equation}
This robust convergence of $k'_t$ to $N_f$ throughout all phases ensures that $M_t$, governed by the rule $M_t = \round{\omega M_{t-1} + (1-\omega) k'_t}$, also converges:
\begin{equation}
\begin{aligned}
    \lim_{t \to \infty} M_t &= \lim_{t \to \infty} \round{\omega M_{t-1} + (1-\omega)k'_t}\\
    &= \round{\omega \lim_{t \to \infty} M_{t-1} + (1-\omega)\lim_{t \to \infty} k'_t}\\
    &= \round{\omega N_f + (1-\omega)N_f}\\
    &= N_f
\end{aligned}
\end{equation}
This convergence of $M_t$ to $N_f$ is critical as it creates a self-reinforcing mechanism: it ensures that free-riders are continuously targeted for selection, whether they are in a reputation growth phase (due to high $\rho_f^t$) or a decline phase (due to large negative $\delta r_f^t$).
\end{proof}

\subsection{Phase 1: Decline of Honest Client Participation Rate}
\label{phase:hc_decline}
By Assumption 2, free-riders $\FC$ submit bids with an enhanced cost-effectiveness ratio $\rho_f^t \approx \gamma \bar{\rho}_h^t$ where $\gamma > 1$. According to Definition 1 (Selection based on Reverse Auction), this ensures that free-riders have a higher selection probability than honest clients $\HC$, i.e., $p_f^t > p_h^t$.

With the selection size $M_t$ converging to $N_f$ (as established in Lemma \ref{lemma:mt_convergence}), and free-riders being preferentially selected, the $M_t \approx N_f$ slots in the selected set $C_t$ will be predominantly filled by free-riders. Consequently, the proportion of honest clients in $C_t$, denoted by $\eta_h^t = \frac{|C_t \cap \HC|}{M_t}$, will diminish over time:
\begin{equation}
    \eta_h^t = \frac{|C_t \cap \HC|}{M_t} \approx \frac{|C_t \cap \HC|}{N_f} \to 0
\end{equation}
This implies that there exists a time $T_1$ such that for all subsequent rounds $t > T_1$, the average honest participation rate over the recent $\tau$ rounds (as relevant for Assumption 2) falls below the critical threshold $\theta$:
\begin{equation}
    \exists T_1 \text{ s.t. } \forall t > T_1, \quad \frac{1}{\tau}\sum_{j=t-\tau}^{t-1} \eta_h^j < \theta
\end{equation}
During the initial part of this phase, while $M_t$ is still stabilizing or if $\eta_h^t$ has not yet dropped significantly, both free-riders and selected honest clients might experience reputation gains. However, the preferential selection ensures free-riders are more consistently chosen, contributing to the conditions for $M_t$'s convergence as described in Lemma \ref{lemma:mt_convergence}.

\subsection{Phase 2: Free-Rider Performance Degradation}
\label{phase:fr_degradation}
Let $T_2 = T_1 + \tau$. The time $T_2$ accounts for the window $\tau$ needed for the reduced honest client participation (established in Section \ref{phase:hc_decline}) to impact the free-riders' ability to generate effective updates.
For rounds $t > T_2$, the condition $\frac{1}{\tau}\sum_{j=t-\tau}^{t-1} \eta_h^j < \theta$ holds. According to Assumption 2 (Free-Rider's Greedy Behavior), the success probability $q_f^t$ for free-riders $f \in \FC$ drops significantly:
\begin{equation}
    \forall f \in \FC, \quad q_f^t \to 0 \quad \text{for } t > T_2
\end{equation}
At this stage, free-riders can no longer reliably mimic honest updates due to the lack of fresh, genuine gradient information from a sufficient number of honest participants in the recent selection history.

\subsection{Phase 3: Ensured Selection and Reputation Collapse of Free-Riders}
\label{phase:fr_collapse_main}
Consider the system dynamics for $t > T_2$. Free-riders now consistently fail when selected ($q_f^t \approx 0$) due to the reasons outlined in Section \ref{phase:fr_degradation}.
Crucially, Lemma \ref{lemma:mt_convergence} established that $M_t \approx N_f$. The mechanism for $M_t$ adjustment relies on $k'_t$, which identifies the group of $N_f$ clients with the most significant absolute reputation changes $|\delta r_i^t|$. As free-riders begin to fail, their reputation $r_f^t$ decreases by $\beta^{k_f^t}$ per failure (Definition 2), leading to large negative values for $\delta r_f^t \approx -\sum \beta^{k_f^j}$. The magnitude of these changes ensures that free-riders continue to dominate the top $N_f$ positions in the sorted list of $|\delta r_i^t|$, thus $k'_t \approx N_f$ is maintained, and $M_t \approx N_f$ persists.
This means the system continues to select these $N_f$ free-riders, who are now trapped in a cycle of being selected and then failing. This persistent selection and failure leads to their reputation collapse, as formalized in Lemma \ref{lemma:reputation_collapse}.

\begin{lemma}[Reputation Collapse of Free-Riders]
\label{lemma:reputation_collapse}
For any free-rider $f \in \mathcal{F}$, there exists a finite time $T_f$ such that its reputation $r_f^{T_f} < r_{thresh}$.
\end{lemma}
\begin{proof}
Consider $t > T_2$, where $q_f^t \approx 0$ for any $f \in \mathcal{F}$ (from Section \ref{phase:fr_degradation}).
The expected change in reputation for a free-rider $f \in \FC$, if selected, is (from Definition 2):
\begin{equation}
    \E[\Delta r_f^t | r_f^t, k_f^t] = p_f^t [q_f^t \alpha - (1-q_f^t)\beta^{k_f^t}]
\end{equation}
Since $q_f^t \approx 0$, this simplifies to:
\begin{equation}
    \E[\Delta r_f^t] \approx - p_f^t \beta^{k_f^t}
\end{equation}
As established in Lemma \ref{lemma:mt_convergence} and discussed in Section \ref{phase:fr_collapse_main}, $M_t \to N_f$, and the selected clients are almost exclusively free-riders. Thus, the selection probability $p_f^t \approx 1$ for any $f \in \mathcal{F}$ (assuming it has not yet been eliminated and is among the $N_f$ "most notable" clients in terms of $|\delta r_i^t|$). Let's assume a lower bound $p_f^t \ge p_{min} > 0$ for free-riders that are candidates for selection within $M_t \approx N_f$ before their reputation drops below $r_{thresh}$.

Let $\Delta r = r_f^{T_2} - r_{thresh}$ be the amount of reputation a free-rider $f$ (active at $T_2$) needs to lose to fall below $r_{thresh}$. We need the sum of expected decreases from $T_2$ until some time $T_f-1$ to exceed $\Delta r$:
\begin{equation}
    \sum_{t=T_2}^{T_f-1} \E[-\Delta r_f^t] \approx \sum_{t=T_2}^{T_f-1} p_f^t \beta^{k_f^t} > \Delta r
\end{equation}
Using the lower bounds $p_{min}$ for $p_f^t$ and $k_{min} = k_f^{T_2}$ for $k_f^t$ (since $k_f^t$ is non-decreasing):
\begin{equation}
    \sum_{t=T_2}^{T_f-1} p_f^t \beta^{k_f^t} \ge p_{min} \sum_{t=T_2}^{T_f-1} \beta^{k_f^t}
\end{equation}
Since $k_f^t$ is non-decreasing and $\beta > 1$, the terms $\beta^{k_f^t}$ are themselves non-decreasing and positive. A simple lower bound for the sum is:
\begin{equation}
    p_{min} \sum_{t=T_2}^{T_f-1} \beta^{k_f^t} \ge p_{min} \beta^{k_{min}} (T_f - T_2)
\end{equation}
To ensure the reputation drops below the threshold $r_{thresh}$, we require:
\begin{equation}
    p_{min} \beta^{k_{min}} (T_f - T_2) > \Delta r
\end{equation}
Solving for $T_f$, we get:
\begin{equation}
    T_f > T_2 + \frac{\Delta r}{p_{min} \beta^{k_{min}}}
\end{equation}
Since $\Delta r$ is finite, $p_{min} > 0$, $\beta > 1$, and $k_{min}$ is finite, the right-hand side is a finite value. Thus, a finite time $T_f$ must exist for each free-rider $f \in \mathcal{F}$ by which its reputation falls below $r_{thresh}$.
\end{proof}

\begin{theorem}[Free-Rider Elimination]
\label{theorem:fr_elimination}
The proposed mechanism ensures that all free-riders are eventually eliminated from the system.
\end{theorem}
\begin{proof}
According to Lemma \ref{lemma:reputation_collapse}, for every free-rider $f \in \mathcal{F}$, there exists a finite time $T_f$ at which its reputation $r_f^{T_f}$ drops below the threshold $r_{thresh}$.
Let $T_{max} = \max_{f \in \mathcal{F}} \{T_f\}$. Since each $T_f$ is finite, their maximum, $T_{max}$, is also finite.
For any time $t > T_{max}$, it holds that for all free-riders $f \in \mathcal{F}$, their reputation $r_f^t < r_{thresh}$.
Participants whose reputation scores are below $r_{thresh}$ are effectively eliminated or ignored by the system's selection process (as they would not be selected or their contributions would be deemed unreliable).
Therefore, after a finite time $T_{max}$, the system contains no active free-riders.
\end{proof}

% Add References Section using BibTeX or Biblatex
% \bibliographystyle{IEEEtran}
% \bibliography{IEEEabrv, your_bib_file}

\end{document}